%%% Template originaly created by Karol Kozioł (mail@karol-koziol.net) and modified for ShareLaTeX use

\documentclass[a4paper,11pt]{article}

\usepackage[T1]{fontenc}
\usepackage[utf8]{inputenc}
\usepackage{graphicx}
\usepackage{xcolor}

\renewcommand\familydefault{\sfdefault}
\usepackage{tgheros}
\usepackage[defaultmono]{droidmono}

\usepackage{amsmath,amssymb,amsthm,textcomp}
\usepackage{enumerate}
\usepackage{multicol}
\usepackage{tikz}

\usepackage{geometry}
\geometry{total={210mm,297mm},
left=25mm,right=25mm,%
bindingoffset=0mm, top=20mm,bottom=20mm}


\linespread{1.3}

\newcommand{\linia}{\rule{\linewidth}{0.5pt}}

% custom theorems if needed
\newtheoremstyle{mytheor}
    {1ex}{1ex}{\normalfont}{0pt}{\scshape}{.}{1ex}
    {{\thmname{#1 }}{\thmnumber{#2}}{\thmnote{ (#3)}}}

\theoremstyle{mytheor}
\newtheorem{defi}{Definition}

% my own titles
\makeatletter
\renewcommand{\maketitle}{
\begin{center}
\vspace{2ex}
{\huge \textsc{\@title}}
\vspace{1ex}
\\
\linia\\
\@author \hfill \@date
\vspace{4ex}
\end{center}
}
\makeatother
%%%

% custom footers and headers
\usepackage{fancyhdr}
\pagestyle{fancy}
\lhead{}
\chead{}
\rhead{}
\lfoot{Assignment \textnumero{} 1}
\cfoot{}
\rfoot{Page \thepage}
\renewcommand{\headrulewidth}{0pt}
\renewcommand{\footrulewidth}{0pt}
%

% code listing settings
\usepackage{listings}
\lstset{
    language=Python,
    basicstyle=\ttfamily\small,
    aboveskip={1.0\baselineskip},
    belowskip={1.0\baselineskip},
    columns=fixed,
    extendedchars=true,
    breaklines=true,
    tabsize=4,
    prebreak=\raisebox{0ex}[0ex][0ex]{\ensuremath{\hookleftarrow}},
    frame=lines,
    showtabs=false,
    showspaces=false,
    showstringspaces=false,
    keywordstyle=\color[rgb]{0.627,0.126,0.941},
    commentstyle=\color[rgb]{0.133,0.545,0.133},
    stringstyle=\color[rgb]{01,0,0},
    numbers=left,
    numberstyle=\small,
    stepnumber=1,
    numbersep=10pt,
    captionpos=t,
    escapeinside={\%*}{*)}
}

%%%----------%%%----------%%%----------%%%----------%%%

\begin{document}

\title{Assignment \textnumero{} 1}

\author{Liheng Gong, lg2848@nyu.edu}

\date{09/06/2017}

\maketitle

\section*{Ex. 1.1}

\begin{lstlisting}[label={list:first},caption=Sample Bash code.]
#! /bin/bash
python stage1.py
echo "Stage I done!"
python stage2.py
echo "Stage II done!"
python stage3.py
echo "Stage III done!"
\end{lstlisting}

With one ring, it is trivial. With two rings, it is easy to handle. When there are three rings, it becomes complicated and brain torturing if we start to think how to move ring 1 instead of ring 3. When it comes to four rings, I cannot even get a clear picture without resorting to recursion.

$$H(4,A,B,C)=\left\{
\begin{aligned}
&H(3,A,C,B) \\
&Move\, Ring\, 4\, from\, A\, to\, B\\
&H(3,C,B,A) \\
\end{aligned}
\right.
$$

$$H(3,A,C,B)=\left\{
\begin{aligned}
&H(2,A,B,C) \\
&Move\, Ring\, 3\, from\, A\, to\, B\\
&H(2,B,C,A) \\
\end{aligned}
\right.
$$

$$H(3,A,B,C)=\left\{
\begin{aligned}
&H(2,A,C,B) \\
&Move\, Ring\, 3\, from\, A\, to\, B\\
&H(2,C,B,A) \\
\end{aligned}
\right.
$$

Even if we use recursion, the steps quickly explodes because the number of steps is exponential.


\section*{Ex. 1.2}
\textbf{Answer to a:} Actually a few weeks ago, I would attack this problem by thinking how should I move the first step, which "naturally" leads me to move the smallest ring, which will eventually lead me to nowhere because the details soon overwhelm me.

\textbf{Answer to b:} To make my life easier I need to figure out what a DTH solver does. A DTH solver "collects" all these disordered rings and "puts" them on the target pole by decreasing size from bottom to top.

So if I have a solver for $n-1$ rings, I will begin by asking the same old question: where should ring $n$ go?

Answer is trivial: ring $n$ should go from the source pole(A or B or C) to the target pole B. In order for that, pole B should be empty(or only has ring $n$) before I move ring $n$, which means pole B should not be the target pole for the solver if ring $n$ is on A or C. Of course the pole on which ring $n$ sits should not be the target pole for the solver either.

After applying the solver, pole B is "cleared", so I can move ring $n$ to pole B, and I can then apply the solver again to move ring 1 through $n-1$ to pole B.

But if ring $n$ is on pole B, then I just use the solver and move ring 1 through $n-1$ to pole B and it is done.

So the following pseudo code is naturally derived from the reasoning process above:

~\\
\textbf{Answer to c:}
\begin{lstlisting}[label={list:second},caption=Disordered Towers Hanoi pseudo code.]
procedure DTH(n, A, B, C);
  if n == 1:
    if ring is on A or on C, move it to B;
    else: do nothing;
  else:
    if ring n on A:
      DTH(n-1, A, C, B);
      move ring n from A to B;
      DTH(n-1, C, B, A);
    elseif ring n on B:
      DTH(n-1, A, B,C);
    else:
      DTH(n-1, B, A, C);
      move ring n from C to B;
      DTH(n-1, A, B, C);
  end if
\end{lstlisting}

\textbf{Answer to d:} Actually when designing procedure DTH for \textbf{Answer c}, I noticed that after calling the solver for n-1 rings, these n-1 rings are placed on a pole and sorted. So if I want to move these n-1 rings again, I don't have to call DTH but only have to call a standard Hanoi Tower procedure ToH(n, A, B, C).

~\\

\begin{lstlisting}[label={list:third},caption=Improved Disordered Towers Hanoi pseudo code.]
procedure IDTH(n, A, B, C);
input: n rings distributed arbitrarily among three poles but stacked legally on each pole
output: n rings on pole B, sorted with larger rings below smaller rings

  if n == 1:
    if ring is on A or on C, move it to B;
    else: do nothing;
  else:
    if ring n on A:
      IDTH(n-1, A, C, B);   {target pole is C}
      move ring n from A to B;
      ToH(n-1, C, B, A);    {target pole is B}
    elseif ring n on B:
      IDTH(n-1, A, B,C);
    else:
      IDTH(n-1, B, A, C);   {target pole is A}
      move ring n from C to B;
      ToH(n-1, A, B, C);    {target pole is B}
  end if
\end{lstlisting}

\section*{Ex. 1.3}

\textbf{Answer to a:} Beginners will probably ask which pole should ring 1(the smallest one) go since most beginners tend to delve into the details instead of thinking and analyzing in an abstract way.

~\\
\noindent\textbf{Answer to b:} A pro would ask: which pole should ring n go because by asking that, we are forced to ask the second question very naturally: which pole should ring 1 to ring n-1 go? Since ring n should go to pole B, ring 1 to ring n-1 should go to the spare rings.

Since there are two spare rings, the next question to ask is: which ring should go to pole D and which rings should go to pole C?
My conclusion is that ring n-1 should go to pole D and ring 1 to n-2 should go to pole C. 

Actually I'm having difficulty to find a strict prove at the moment, but I still want to write down my thoughts since I believe they're very close to a strict proof.

~\\
First, let's write the pseudo code and calculate the movement steps. If we move n-1 from A to D, the operation sequence is:
\begin{itemize}
    \item move ring 1 to n-2 from A to C
    \item move ring n-1 from A to D
    \item move ring n from A to B
    \item move ring n-1 from D to B
    \item move ring 1 to n -2 from C to B
\end{itemize}

And the pseudo code is:
\begin{lstlisting}[label={list:fourth},caption=steps listing -- Towers of Hanoi with four poles.]
ToH4Poles(n, A, B, C, D)
    if n == 1:
      move ring 1 from A to B
    elseif n == 2:
      move ring 1 from A to D
      move ring 2 from A to B
      move ring 1 from D to B
    else:
      ToH4Poles(n-2, A, C, B, D);
      move ring n-1 from A to D
      move ring n from A to B
      move ring n-1 from D to B
      ToH4Poles(n-2, C, B, A, D);
\end{lstlisting}

and from the recursion process it is easy to know the total steps are:

$$
R(n) = 2R(n-2) + 3
$$
This non-homogeneous recursion equation can be easily solved(using characteristic equation) as below:
$$
R(n) = \frac{3+2\sqrt{2}}{2}\cdot \sqrt{2}^{n}+\frac{3-2\sqrt{2}}{2}\cdot (-\sqrt{2})^{n}-3
$$
Let's re-arrange the above expression to a more readable form:
$$R(n)=\left\{
\begin{aligned}
&2^{k+2}-3 & if\; &n=2k+1 \\
&3\cdot 2^{k} - 3  & if\; &  n =2k \\
\end{aligned}
\right.
$$


\vspace{1 ex}
\noindent Now suppose some plate other than n-1, say k, is placed on pole D. 
Then the sequence is:
\begin{itemize}
    \item \textcolor{red}{ring 1 to k-1 from A to B}
    \item move ring k from A to D
    \item move ring k+1 to n-1 from A to C
    \item \textcolor{red}{move ring 1 to k-1 from B to C}
    \item move ring n from A to B
    \item \textcolor{red}{move ring 1 to k-1 from C to A}
    \item move ring k+1 to n-1 from C to B
    \item move ring k from D to B
    \item \textcolor{red}{move ring 1 to k-1 from A to B}
\end{itemize}


The four steps marked \textcolor{red}{red} above are actually standard Hanoi Tower problem because there are only three poles available to use when these movements occur.
\iffalse
So we can write down the total steps of n rings as:
\begin{equation}
\begin{split}M(n) = &M(k-1) + 1 + 2^{n-k-2} - 1 + 1 + 2^{k-2} -1 + 1 + 2^{k-2}-1 + 2^{n-k-2} - 1 + 1 + M(k-1) \\
&=2M(k-1) + 2^{n-k-1} + 2^{k-1} -1
\end{split}
\end{equation}
\fi

We can also see from the above steps that there are redundant moves in the above procedure; as a result, they are slower than the procedure ToH4Poles(n, A, B, C, D).


\section*{Ex. 1.4}
\textbf{Answer to a:} Actually the hint is enough to solve this question. To facilitate the discussion, I'll list the pseudo code of standard Hanoi problem:


\begin{lstlisting}[label={list:seventh},caption=Standard Tower of Hanoi procedure.]
procedure ToH(n, A, B, C);
  if n == 1:
    move ring 1 from A to B;
  else:
    ToH(n-1, A, C, B);
    move ring n from A to B;
    ToH(n-1, C, B, A);
  end if
\end{lstlisting}
We can see from the procedure above that for rings 1 through $n-1$ when they are moved from pole A to pole C, the number of flips for each ring is the same as the number of flips when they are moved from pole C to pole B, which means each of these rings is flipped even times. So the answer is: for ring 1 through $n-1$, the color is red.

\noindent And because ring n is only flipped once, its color is reversed and hence is white.

~\\
\textbf{Answer to b:} The only difference between b) and a) is the color of ring $n$. So same trick here: let's ask the most important question: where ring $n$ should go if we want to make sure that its color is red when it is placed to pole B? The answer is trivial: ring $n$ should be flipped even times and since the minimum times is 2, so ring $n$ should be flipped twice before it lands on pole B, which requires its moving sequence to be: from pole A to pole C, then from pole C to pole B. The moving sequence of ring $n$ "forces" the moving sequence of stack of ring 1 through n-1 to be:
\begin{itemize}
    \item move ring 1 through n-1 from A to B
    \item move ring n from A to C
    \item move ring 1 through n-1 from B to A
    \item move ring n from C to B
    \item move ring 1 through n-1 from A to B
\end{itemize}
But wait a moment, although the color of ring n is red now, how can we make sure that the stack of ring 1 through n is also red? Note that this stack as a whole is "moved" three times. How about add another "move" for this stack to make sure it is "moved" even times?
\begin{itemize}
    \item move ring 1 through n-1 from A to B
    \item move ring n from A to C
    \item move ring 1 through n-1 from B to A
    \item move ring n from C to B
    \item move ring 1 through n-1 from A to C
    \item move ring 1 through n-1 from C to B
\end{itemize}
And from the above discussion I have my first solution:
\begin{lstlisting}[label={list:seventh0},caption=Red Tower of Hanoi procedure.]
procedure RTH(n, A, B, C);
  if n == 1:
    move ring 1 from A to C;
    move ring 1 from C to B;
  else:
    RTH(n-1, A, B, C);
    move ring n from A to C;
    RTH(n-1, B, A, C);
    move ring n from C to B;
    RTH(n-1, A, C, B);
    RTH(n-1, C, B, A);
  end if
\end{lstlisting}
But after attending the tutoring session hosted by Jianyang, I figured out that I needed to revise my solution. I realized that I was still focusing too much on the details when working out the above solution. I need to follow Professor Siegel's advice to think abstractly, i.e. to figure out what my procedure should do instead of focusing on what one specific step in my function does. Now let's redesign the function from scratch and focus on what the function does as a whole. So the input is: n sorted rings on pole A with all the top sides red. And the output should be: $n$ sorted rings on pole B with all the top sides red

As a beginner, to make my life easier, I'll steal Professor Siegel's trick by imagining a solver for $n-1$ rings. The first question to ask is exactly the same: where should ring $n$ go? Same answer: it should first go to pole C, then go to pole D. And with $n$ rings my function should do things as below:
\begin{itemize}
    \item use the solver to move ring 1 through n-1 from A to B
    \item move ring n from A to C
    \item use the solver to move ring 1 through n-1 from B to A
    \item move ring n from C to B
    \item use the solver to move ring 1 through n-1 from A to B
\end{itemize}
How do we know the function above does what it is expected to do?
First, after each "move", the solver ensures that all the top sides of ring 1 through n-1 are red. Second, ring n is flipped twice, so its top side is also red. So, our function is correct.

The steps above can be easily translated to pseudo code:
\begin{lstlisting}[label={list:seventh1},caption=Improved Red Tower of Hanoi procedure.]
procedure IRTH(n, A, B, C);
  if n == 1:
    move ring 1 from A to C;
    move ring 1 from C to B;
  else:
    IRTH(n-1, A, B, C);
    move ring n from A to C;
    IRTH(n-1, B, A, C);
    move ring n from C to B;
    IRTH(n-1, A, B, C);
\end{lstlisting}
IRTH is definitely faster than RTH because there are 3 recursive calls in the IRTH versus 4 recursive calls in RTH.

\textbf{Answer to c:} 


\section*{Ex. 1.5}
test test
\begin{lstlisting}[label={list:eighth},caption=Sample Bash code.]
#! /bin/bash
test test
\end{lstlisting}

\section*{Ex. 1.6}
test test
\begin{lstlisting}[label={list:nineth},caption=Sample Bash code.]
#! /bin/bash
test test
\end{lstlisting}


\section*{Ex. 1.7}
test test
\begin{lstlisting}[label={list:tenth},caption=Sample Bash code.]
#! /bin/bash
test test
\end{lstlisting}



\section*{Ex. 1.12}

To be honest, I highly doubted whether the algorithm provided by Professor Siegel is the correct one(i.e. whether it is the slowest one) and I even "designed" a slower one(at least I though it was slower), but after some thought, I started to realize how stupid I am.

First the algorithm by Professor Siegel:

\begin{lstlisting}[label={list:eleventh},caption=Pseudo code -- slow Towers of Hanoi.]

procedure SloTh(n, A, B, C);
   input: the n smallest rings start on pole A in sorted order;
   output: the n rings are moved in a legal, complicated pattern that ends with all on B in sorted order and as slow as possible;
   if n equals 1 then:
     move ring 1 from A to C;
     move ring 1 from C to B;
   else:
     SloTh(n - 1, A, B, C); # move ring 1:n-1 from pole A to pole B
     move ring n from A to C;
     SloTh(n - 1, B, A, C); # move ring 1:n-1 from pole B to pole A
     move ring n from C to B;
     SloTh(n - 1, A, B, C); # move ring 1:n-1 from pole A to pole B
   endif
end_ SloTh

\end{lstlisting}

[The following discussion were based on my own understanding on September 5th. However, Professor Siegel explained this problem thoroughly in the recitation session on September 9th. On one hand I was happy that my own understanding was correct, but one the other hand it is basically the same with Professor Seigel's explanation.]

Discussion regarding Listing~\ref{list:third}\ldots{} 
As professor Seigel said, the first step to solve a problem is to ask the right question. So the first question I want to ask is: for n rings how many configurations are possible?

\textbf{Answer}: The constraint is that smaller rings should be placed on top of larger rings. So let's place these n rings one by one from large to small to ensure of that constraint. As discussed, the possible places for ring n(the largest one) is 3, and the possible places for ring n - 1 is 3, and so on. As a result, for n rings, the number of all configurations Cf(n) is:

$$Cf(n) = 3^{n}$$

Now let's calculate the ring-move count R(n) for procedure SloTh(n, A, B, C).
From the recursion we know that:
$$R(n)=\left\{
\begin{aligned}
&2  & if\; &  n = 1 \\
&3R(n-1) + 2  & if\; &  n > 1 \\
\end{aligned}
\right.
$$
The above equation is a linear recursive equation and we can easily find its solution as:
$$
R(n) = 3^{n} - 1
$$
Each ring remove produces one configuration, and if we count the initial configuration(i.e. all n rings are on pole A), then the number of all possible configurations from these moves are:
$$
Cm(n) = 3^{n}
$$
Obviously, $$Cf(n) = Cm(n)$$, but can we say that procedure SloTh(n, A, B, C) produces all the possible configurations? Unfortunately we do not have much confidence to say yes at the moment because there might be duplicates in configurations produced by procedure SloTh(n, A, B, C).

So our next task is to prove(or disprove) all configurations produced by procedure SloTh(n, A, B, C) are unique, i.e. there are no duplicates in them.

The only thing we can rely on is the recursion relationship.

First, let's check the base case(i.e. n = 1)
When n = 1, we move ring 1 from pole A to pole C, then from pole C to pole B. This produces two steps and 3 different configurations. So our guess is correct for n = 1.

Now let's assume that SloTh(n - 1, A, B, C) produces $3^{n-1}$ different configurations(i.e. no duplicates). It is time to stare at the recursive procedure.

When SloTh(n - 1, A, B, C) is called, the largest ring(ring n) sits on pole A, so we have $3^{n-1}$ different configurations with ring n on pole A.

Then our program moves ring n from pole A to pole C and calls SloTh(n - 1, B, A, C). Same story, we have $3^{n-1}$ different configurations with ring n on pole C.

At last, our program moves ring n from pole C to pole B and calls Sloth(n - 1, A, B, C), this produces $3^{n-1}$ different configurations with ring n on pole B.

There are no duplicates in the configurations above because:

1. When ring n sit on a specific pole(A or B or C), there are no duplicates.

2. When ring n sit on different poles, there are still no duplicates because ring n's place is different.




\end{document}
