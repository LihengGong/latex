%%% Template originaly created by Karol Kozioł (mail@karol-koziol.net) and modified for ShareLaTeX use

\documentclass[a4paper,11pt]{article}

\usepackage[T1]{fontenc}
\usepackage[utf8]{inputenc}
\usepackage{graphicx}
\usepackage{xcolor}

\renewcommand\familydefault{\sfdefault}
\usepackage{tgheros}
\usepackage[defaultmono]{droidmono}

\usepackage{amsmath,amssymb,amsthm,textcomp}
\usepackage{enumerate}
\usepackage{multicol}
\usepackage{tikz}
\usepackage{algpseudocode}
\usepackage{algorithm}

\usepackage{geometry}
\geometry{total={210mm,297mm},
left=25mm,right=25mm,%
bindingoffset=0mm, top=20mm,bottom=20mm}


\linespread{1.3}

\newcommand{\linia}{\rule{\linewidth}{0.5pt}}


\algnewcommand\algorithmicforeach{\textbf{for each}}
\algdef{S}[FOR]{ForEach}[1]{\algorithmicforeach\ #1\ \algorithmicdo}

%\algdef{S}{IN}[1]{\algorithmicin\ #1}

% custom theorems if needed
\newtheoremstyle{mytheor}
    {1ex}{1ex}{\normalfont}{0pt}{\scshape}{.}{1ex}
    {{\thmname{#1 }}{\thmnumber{#2}}{\thmnote{ (#3)}}}

\theoremstyle{mytheor}
\newtheorem{defi}{Definition}

% my own titles
\makeatletter
\renewcommand{\maketitle}{
\begin{center}
\vspace{2ex}
{\huge \textsc{\@title}}
\vspace{1ex}
\\
\linia\\
\@author \hfill \@date
\vspace{4ex}
\end{center}
}
\makeatother
%%%

% custom footers and headers
\usepackage{fancyhdr}
\pagestyle{fancy}
\lhead{}
\chead{}
\rhead{}
\lfoot{Assignment \textnumero{} 2}
\cfoot{}
\rfoot{Page \thepage}
\renewcommand{\headrulewidth}{0pt}
\renewcommand{\footrulewidth}{0pt}
%

% code listing settings
\usepackage{listings}
\lstset{
    language=Python,
    basicstyle=\ttfamily\small,
    aboveskip={1.0\baselineskip},
    belowskip={1.0\baselineskip},
    columns=fixed,
    extendedchars=true,
    breaklines=true,
    tabsize=4,
    prebreak=\raisebox{0ex}[0ex][0ex]{\ensuremath{\hookleftarrow}},
    frame=lines,
    showtabs=false,
    showspaces=false,
    showstringspaces=false,
    keywordstyle=\color[rgb]{0.627,0.126,0.941},
    commentstyle=\color[rgb]{0.133,0.545,0.133},
    stringstyle=\color[rgb]{01,0,0},
    numbers=left,
    numberstyle=\small,
    stepnumber=1,
    numbersep=10pt,
    captionpos=t,
    escapeinside={\%*}{*)}
}

%%%----------%%%----------%%%----------%%%----------%%%

\begin{document}

\title{Assignment \textnumero{} 2}

\author{Liheng Gong, lg2848@nyu.edu}

\date{09/14/2017}

\maketitle

\section*{Ex. 2.1}
\noindent\textbf{Answer to a:} preorder: visit root first, then recursively visit left subtree and right subtree. So the list should be: \\51, 6, 3, 5, 14, 7, 11
\vspace{1.2in}


\noindent\textbf{Answer to b:} inorder: (recursively) visit left subtree first, then visit root, then (recursively) visit right subtree. So the list should be:\\ 3, 5, 6, 7, 11, 14, 51 \\Due to the properties of binary search tree(BST), inorder traversal of a BST should produce a sorted list of numbers in ascending order.
\vspace{1.2in}

\noindent\textbf{Answer to b:} postorder: (recursively) visit left subtree and right subtree, then visit root node. So the list should be: \\5, 3, 11, 7, 14, 6, 51

\vspace{1.2in}

\section*{Ex. 2.30}
No matter how the representation varies, the structure of a post order traversal procedure is the same. The node's each child is recursively visited, and then the node itself is visited upon exit.
\begin{algorithm}[H]
\caption{Post order traversal of a specific representation of an arbitrary tree}\label{POSTORDER_2_30}
\begin{algorithmic}[1]
\State global $child[m,n]$
\Procedure{PostOrder}{$t$} \Comment{$PostOrder$ receives vertex name as parameter}
  \ForEach {$i$ in $range[1 ... child[t, 0]]$} \Comment{$child[t,0]$ is the number of child for t}
    \State PostOrder($child[t, i]$) \Comment{$child[t,i]$ is the name of child for t}
  \EndFor
  \State visit $t$ \Comment{Visit $t$ upon exit}
\EndProcedure
\end{algorithmic}
\end{algorithm}

\vspace{1.2in}

\section*{Ex.2.2}
Since each internal node is an operator, to evaluate the value using this operator, we need to first get the value of both of its children, which implies a post order traversal.

\begin{algorithm}[H]
\caption{Tree evaluation of an arithmatic expression}\label{TREEEVALUATION_2_2}
\begin{algorithmic}[1]
\Function{TreeEval}{$T$}
  \If {$T$ is leaf}
    \State return $T.val$
  \EndIf
  \State leftVal = $TreeEval(T.left)$
  \State rightVal = $TreeEval(T.right)$
  \State return $Eval(T.op, leftVal, rightVal)$
\EndFunction
\end{algorithmic}
\end{algorithm}

\vspace{1.2in}

\section*{Ex. 2.3}
\noindent\textbf{Answer to a):} By definition, if $x$ is a tree vetex, $x.small$ is expected to store the smallest value in the subtree rooted by $x$(including $x$ itself.) after the evaluation. So let's reason out the evaluation process.

For a specific node $t$, if $t$ has no children(i.e. $t$ is a leaf), then $t.small \gets t.val$. Otherwise, we need to traverse every child node, say $w$, of $t$ and get $w.small$. And this process can be done recursively. After all of $t$'s children's $.small$ field have been updated, $t.small$ can be updated by comparing those values and selecting the smallest one. The update of $t.small$ can only be done in a postorder fashion because $t.small$ cannot be decided until all of its children's $.small$ field have been updated.

The above process can be translated to pseudo code below:

\begin{algorithm}[H]
\caption{Questionable tree evaluation of an arithmetic expression}\label{sec:TREEEVALUATION_2_3_1}
\begin{algorithmic}[1]
\Function{TreeMinVal}{$T$}
  \If{T is a leaf}
    \State $T.small \gets T.val$
    \State return $T.small$
  \Else
    \State $minVal \gets T.val$
    \ForEach{child $w$ of $T$}
      \State $minVal \gets min\{minVal, TreeMinVal(w)$\} \Comment{Recursively update all of $t$'s children's $.small$ field}
    \EndFor
    \State $T.small \gets minVal$
    \State return $T.small$
  \EndIf
\EndFunction
\end{algorithmic}
\end{algorithm}

After some analysis, the code in Algorithm \ref{sec:TREEEVALUATION_2_3_1} can be made more compact as below:

\begin{algorithm}[H]
\caption{Questionable tree evaluation of an arithmetic expression}\label{sec:TREEEVALUATION_2_3_2}
\begin{algorithmic}[1]
\Function{TreeMinVal2}{$T$}
  \State $minVal \gets T.val$
  \ForEach{child $w$ of $T$}
    \State $minVal \gets min\{minVal, TreeMinVal2(w)$\} \Comment{Recursively update all of $t$'s children's $.small$ field}
  \EndFor
  \State $T.small \gets minVal$
  \State return $T.small$
\EndFunction
\end{algorithmic}
\end{algorithm}

\vspace{1.2in}

\noindent\textbf{Answer to b):} In addition to the smallest value, we also store the address of that node. Also, in addition to returning the smallest value, we return a tuple this time, including both the smallest value and the address of the corresponding node.

\begin{algorithm}[H]
\caption{Questionable tree evaluation of an arithmetic expression}\label{sec:TREEEVALUATION_2_3_b}
\begin{algorithmic}[1]
\Function{TreeMinValPointer}{$T$}
  \State $minVal \gets T.val$
  \State $minPtr \gets T$
  \ForEach{child $w$ of $T$}
    \If{$.value\;of\;TreeMinValPointer(w) < minVal$}
      \State $minVal \gets .value\;of\;TreeMinValPointer(w)$
      \State $minPtr \gets .pointer\;of\;TreeMinValPointer(w)$
    \EndIf
  \EndFor
  \State $T.small \gets minVal$
  \State $T.which \gets minPtr$
  \State return ($T.small, T.which$)   \Comment{return both value and pointer}
\EndFunction
\end{algorithmic}
\end{algorithm}

\vspace{1.2in}

\section*{Ex. 2.4}


\end{document}
